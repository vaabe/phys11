\documentclass[11pt]{extarticle}
\usepackage[margin=2.0cm]{geometry}
\usepackage[UKenglish]{babel}
\usepackage{amssymb}
\usepackage{amsmath}
\usepackage{graphicx}
\graphicspath{{./pics/}}
\usepackage{hyperref}
\usepackage{changepage}

\begin{document}

\setlength{\parindent}{0pt}

\begin{center}
	\huge
	Tips for Writing Lab Reports

	\vspace{0.5cm}
\end{center}

Here are some pointers for what your report should generally include:

\hfill 

\begin{itemize}
	\item {\bf Introduction:} a brief overview of the experiments you performed. 

	\item {\bf Theory:} the theory/principles being observed in the experiment. Some of the theory is introduced in the lab manual, but you should do your own research and write it up in your own words. 

	\item {\bf Experiment Setup/Procedure:} what you were trying to measure and how you did it, e.g. the apparatus you used, measuring techniques, etc. Diagrams of the setup are useful. 

	\item {\bf Results:} your data from the experiment. Include any graphs, tables and calculations. 

	\item {\bf Error Analysis:} this involves determining the uncertainty in your measurements. There are different uncertainty metrics you can calculate, depending on the kind of data you have: 
		\begin{itemize}
			\item if you have a single measurement, the uncertainty will usually be in the last recorded significant figure [e.g. a measurement of 7.12 m might have an uncertainty of $\pm$ 0.01 m]. 
			\item if you have an array of data, you can calculate summary statistics, such as the mean or standard deviation. 
			\item note the ``error analysis" doesn't need to be a separate section in your report. You should just include the relevant uncertainty for each piece of data you have. You can find a detailed guide on error analysis in the folder. Read through this and apply the techniques to your data. 
		\end{itemize}

	\item {\bf Discussion:} discuss what your results show, and whether or not you got what you expected. Some discussion points might be:
		\begin{itemize}
			\item sources of error in the experiment, and how they may have compromised the validity of the results 
			\item issues you noticed with the experiment design that may have hindered your ability to get accurate results 
			\item how the uncertainties in the data may have affected the results 
			\item if applicable, you could compare your results to theoretical values and account for any discrepancies you find 
			\item suggestions for how the experiment could be improved 
		\end{itemize}

	\item {\bf Conclusion.} 
\end{itemize}

\hfill 

Each lab manual also has a number of questions, which are highlighted in bold throughout the manual. You should integrate your answers to these questions into your report, wherever you see fit. \\ 

Here are some general tips for structuring your report:

\hfill 

\begin{itemize}
	\item Most of these labs have several experiments. Try to treat each experiment separately, rather than mushing all your content together. E.g. if a lab has two experiments, \textit{Testing Newton's 2nd Law} and \textit{Measuring Acceleration Due To Gravity}, then for \textit{each} experiment you should:
	\begin{itemize}
		\item describe its purpose (what are you trying to do/measure/determine?)
		\item describe the general setup (how you went about doing/measuring this)
		\item give your results and uncertainties 
		\item discuss. 
	\end{itemize}

	\item Try to write your report in a ``linear" order. This follows from the previous point. Many of you tend to introduce all the experiments in one huge block, then throw all the results in another huge block, i.e. you're partitioning your report by content type, rather than subject. This makes for an unwieldy reading experience. If you introduce Experiment 1, and then go on to talk about something else before continuing with Experiment 1, by the time you return to it [perhaps in the ``results" section] I'll have forgotten all about it. It would [surely?] be better to do it the other way round, i.e. put everything for one experiment (intro/setup/results/discuss) in one place, \textit{then} move on to the next experiment, etc. Otherwise I have to constantly flick back and forth between page 2 and page 7 just to understand what you're talking about [and surely you do too, as the writer]. Doing stuff linearly is easier to read, and easier to write.  

	\item Try not to unnecessarily regurgitate the lab manual. Some of you are including stuff like ``Familiarize yourself with Capstone" or ``drag the position icon into the display"---these are instructions for people doing the lab. They aren't relevant to your report. Again, use your intuition. 

	\item Make your report self-contained. E.g. if you reference a graph, put it in your report. Don't expect me to have memorized what ``graph 3b" is. 

	\item The bold questions in the lab manual. Some of you are tacking your answers to the end of your report. This makes sense if it's a stand-alone question that doesn't seem to fit in anywhere else in the report. But if the question is an essential part of the discussion of a particular experiment, you'd do well to integrate your answer into the discussion, rather than leave me waiting [with bated breath] until the end.  
\end{itemize} 

\hfill 

Ultimately it's up to you how you structure your report. Don't feel you need to adhere rigidly to the outline above. They're only guidelines, not rules, and different labs will have different requirements. You may decide to merge certain sections to make your report flow better. Use your intuition. Whatever you choose to do, make sure it's presented clearly. \\ 


\end{document}
