\documentclass[11pt]{extarticle}
\usepackage[margin=2.5cm]{geometry}
\usepackage[UKenglish]{babel}
\usepackage{amssymb}
\usepackage{amsmath}
\usepackage{graphicx}
\graphicspath{{./pics/}}
\usepackage{hyperref}
\usepackage{changepage}
\usepackage{mathpazo}

\begin{document}

\setlength{\parindent}{0pt}

\begin{center}
	\huge 
	GP I Lab

	\vspace{0.5cm}
	\large 
	Fall 2020
\end{center}

\begin{adjustwidth}{0.5cm}{}
	instructor: Vinu Abeywick \\ 
	email: \href{mailto:vabe@nyu.edu}{\texttt{vabe@nyu.edu}} \\ 
	section 10: Wed 4.00-6.20pm \\ 
	section 20: Thu 11.00am-1.20pm \\ 
	location: Meyer 222 \\ 
	zoom office hours: Wed 1-2pm, Fri 2-3pm \\  
	web: \url{https://physics.nyu.edu/~physlab/GenPhysI_PhysII/genphys1.html}
\end{adjustwidth}

\hfill 

{\bf \Large Description} \\ 

This course is designed to demonstrate basic principles of mechanics in a laboratory setting. All labs are offered both in-person and online. You decide at the beginning of the semester which mode you will take. You can find all the lab manuals at the course website linked above. \\ 

{\bf In-person students:} you will perform the experiments individually. Please show up to the classroom at the scheduled time and bring a mask. You must wear your mask and use the same lab bench throughout the semester. We'll also provide safety goggles and gloves for you to use if you wish. You should read the manual before the session so you know what to expect. You will write your lab report based on the experiment you perform. \\ 

{\bf Online students:} an online folder with the lab materials will be shared with you. Each week we'll post a video of the lab manager performing the lab along with the data. You will write your lab report based on the videos and data provided. There will be no regular meetings for the online lab---it's your responsibility to watch the videos each week and keep up with the schedule.You can get help from the instructor during the zoom office hours. \\  

{\bf \Large Schedule} \\ 

\begin{table}[h]
	\centering
		\begin{tabular}{ c 	c }
			Sep 16-17 & Policy Review \\ 
			Sep 23-24 & Motion 1 \\ 
			Sep 30-Oct 1 & Motion 2 \\ 
			Oct 7-8 & Newton's 2nd Law \\ 
			Oct 14-15 & Work-Energy \\ 
			Oct 21-22 & Conservation of Energy \\ 
			Oct 28-29 & Collisions in 1D \\ 
			Nov 4-5 & Ballistic Pendulum \\ 
			Nov 11-12 & Centripetal Force \\ 
			Nov 18-19 & Human Arm \\ 
			Nov 25-26 & [No Labs] \\ 
			Dec 2-3 & Absolute Zero \\ 
		\end{tabular}
\end{table}

\newpage

{\bf \Large Attendance} \\ 

Obviously, attendance is required for in-person labs. If you need to be absent please email me in advance. For online labs there is no attendance requirement. The online office hour is optional and anyone can show up to ask questions. \\ 

{\bf \Large Lab Reports} \\ 

Each week you will write a lab report on the experiment you performed that week. The report is due one week after the scheduled lab session. In the shared folder we've posted some resources on how to write lab reports and perform error analysis on your data. If you're new to report-writing, you may want to read these over. But you don't need to adhere to them strictly. Different labs will have different requirements, and you should use your intuition for what to include in your report. \\ 

Reports will be graded out of 10. One point will be reserved as a bonus for people who do something beyond what's asked in the lab manual. \\ 

Please email me your reports in pdf format. \\ 

\end{document}
